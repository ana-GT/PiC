\RequirePackage{flashmovie}
\documentclass[10pt]{beamer}
\usepackage{amsmath}
\usepackage{amsfonts}
\usepackage{amssymb}
\usepackage{graphicx} 
\usepackage{subfigure}
\usepackage{float}
\usepackage[spanish]{babel}
\selectlanguage{spanish}
\usepackage[utf8]{inputenc}
\usepackage{hyperref}
\usepackage{color}
\usecolortheme[RGB={0,0,255}]{structure} 
\usetheme{Warsaw}
\setbeamertemplate{subsection in head/foot shaded}[default][20] 
\logo{\includegraphics[height=1.5cm]{figures/PiCLogo.jpg}} 
\author{Peruanas in Computing}
\title{Post-grado en el extranjero: Porqué y como}
\subtitle{We are here with a few slides, You ask the questions!}
\begin{document}
\maketitle

% -------------------------------
% Frame
\begin{frame}

\frametitle{Quiénes somos}
\begin{columns}[t]
\column{0.7 \textwidth}
\begin{itemize}
\item{{\color{blue}{Rosalva Gallardo Valencia}}
\begin{itemize}
  \item{Pontificia Universidad Católica del Perú}
  \item{University of California Irvine}
\end{itemize}
}
\item{{\color{blue}{Meridangela Gutiérrez Jhong}}
\begin{itemize}
  \item{Rochester Institute of Technology}
\end{itemize}
}
\item{{\color{blue}{Ana Huamán Quispe}}
\begin{itemize}
  \item{Universidad Nacional de Ingeniería}
  \item{Georgia Institute of Technology}
\end{itemize}
}
\item{{\color{magenta}{Rosa Enciso (Skype)}}
\begin{itemize}
  \item{Universidad San Antonio de Abad}
  \item{Microsoft}
\end{itemize}
}
\item{{\color{magenta}{Natalie Gil (Skype)}}
\begin{itemize}
  \item{Universidad de Lima}
  \item{Golden Sachs}  
\end{itemize}
}
\item{{\color{magenta}{Yesenia Yari (Skype)}}
\begin{itemize}
  \item{Universidad de }
  \item{Universidad Federal de Rio}  
\end{itemize}
}
\end{itemize}
\column{0.3 \textwidth}
\begin{figure}[h]
			\centering
			  %\subfigure[]{
			  \includegraphics[scale=0.2]{figures/UCI_Logo.jpg} 
	          \label{fig:UCILogo}       
              %}
              %\subfigure[]{
	          \includegraphics[scale=0.2]{figures/GaTech_Logo.jpg} 	
	          \label{fig:GaTechLogo}
              %}
              %\subfigure[]{
	          \includegraphics[scale=0.3]{figures/RIT_Logo.jpg} 	
	          \label{fig:RITLogo}
              %}              
			  %\subfigure[]{
			  \includegraphics[scale=0.2]{figures/UFR_Logo.jpg} 
	          \label{fig:UFRLogo}             
              %}
              %\subfigure[]{
	          \includegraphics[scale=0.2]{figures/Microsoft_Logo.jpg} 	
	          \label{fig:MicrosoftLogo}
              %} 
              %\subfigure[]{
	          \includegraphics[scale=0.15]{figures/Goldman_Logo.jpg} 	
	          \label{fig:GoldmanLogo}
              %}              
            %\caption{}
            \label{fig:Universidades}
	\end{figure} 
\end{columns}

\end{frame}
% -------------------------------

% -------------------------------
\begin{frame}
\frametitle{De qué hablaremos hoy ?}
\begin{enumerate}
\item{Por qué deberías hacer un postgrado en el extranjero?
	\begin{itemize}
		\item{MS o PhD?}
	\end{itemize}
}
\item{Por qué {\color{magenta}{NO}} deberias hacer un postgrado en el extranjero?}
\item{ Decidí postular: Qué pasos debo tomar?}
\end{enumerate}

\end{frame}

%%%%%%%%%%%%%%%%%%%%%%%%%%%%%%%%%%%%%%%%
\begin{frame}
\frametitle{Por qué debería hacer un postgrado en el extranjero ?}
\begin{enumerate}
\item{ {\color{blue}{Aspecto Académico}}
   \begin{itemize}
   \item{Oportunidad de participar en proyectos de investigacion con financiamiento}
   \item{Trabajar con los principales investigadores del mundo}
   \item{Asistir a conferencias, workshops internacionales o eventos organizados por diferentes empresas (i.e. Google)}
   \item{Experimentar un sistema educativo diferente}
   \item{Intercambiar ideas con estudiantes de todas partes del mundo}
   \end{itemize}
}
\item{ {\color{blue}{Aspecto Cultural}}
    \begin{itemize}
	\item{ Aprender sobre otras culturas}
	\item{ Dar a conocer el Perú}
	\item{ Aprender otros idiomas}
	\end{itemize}
}	
\item{ {\color{blue}{Oportunidades a Futuro}}
	\begin{itemize}
	\item{Mejores herramientas para ayudar al desarrollo del país}
	\item{Y naturalmente, mejores perspectivas laborales}
	\end{itemize}
}	
\end{enumerate}

\end{frame}

%%%%%%%%%%%%%%%%%%%%%%%%%%%%%%%%%%%%%%%%
\begin{frame}
\frametitle{Por qué {\color{magenta}{NO}} deberías hacer un postgrado en el extranjero ?}
\begin{enumerate}
\item{Porque la situación económica es difícil y prefiero ser estudiante unos years más}
\item{Porque todos lo hacen}
\item{Porque no sé que más hacer}
\item{Porque quiero salir del país}
\end{enumerate}

{\color{magenta}{Punchline:}} Postgrado es mucho más trabajo que el requerido para ser bachiller. Aparte, requiere mucha motivación personal, donde las notas pasan a un segundo plano. Si tu motivo es alguno de los mencionados arriba, piénsalo dos veces
\end{frame}

% -------------------------------
\begin{frame}
\frametitle{ToDo List}
Postular es {\color{magenta}trabajoso}, así que tómate tu tiempo:
\begin{enumerate}
\item{Qué Universidades tienen programas que son de mi interés}
\item{Contribuye a proyectos de Software Libre (Open Source)}
\item{Intégrate a la comunidad científica: Organizaciones Estudiantiles}
\end{enumerate}

\end{frame}

% -------------------------------
\begin{frame}
\frametitle{Requisitos}

%%%%%%%%%%%%%%%%%%%%%%
\begin{columns}[t]

\column{0.7 \textwidth}
Varían entre Universidades, pero por lo general:

\begin{enumerate}
\item{Test of English as a Foreign Language (TOEFL)}
\item{Graduate Record Examination (GRE)}
\item{Statement of Purpose (SoP)}
\item{3 Letters of Recommendation (LoR)}
\item{College Transcripts and Bachelor Diploma}
\end{enumerate}

\begin{figure}
\includegraphics[scale=0.3]{figures/SoP_Logo.jpg} 	
	          \label{fig:SoPLogo}
	          \end{figure}
%%%%%%%%%%%%%%%%%%%%%%%
\column{0.3 \textwidth}
\begin{figure}[h]
			\centering

			  \includegraphics[scale=0.3]{figures/TOEFL_Logo.jpg} 
	          \label{fig:TOEFLLogo}             

	          \includegraphics[scale=0.6]{figures/GRE_Logo.jpg} 	
	          \label{fig:GRELogo}

            \label{fig:Requirements}
	\end{figure} 

\end{columns}

\end{frame}


\end{document}